\documentclass[conference,compsoc]{IEEEtran}
 
\def\code#1{\texttt{#1}}

\usepackage{lipsum}

\usepackage{datetime}
\usepackage{tikz}
\usepackage{graphicx}
\graphicspath{ {./images/} }

% *** CITATION PACKAGES ***
%
\ifCLASSOPTIONcompsoc
  % IEEE Computer Society needs nocompress option
  % requires cite.sty v4.0 or later (November 2003)
  \usepackage[nocompress]{cite}
\else
  % normal IEEE
  \usepackage{cite} 
\fi 
% cite.sty was written by Donald Arseneau
% V1.6 and later of IEEEtran pre-defines the format of the cite.sty package
% \cite{} output to follow that of the IEEE. Loading the cite package will
% result in citation numbers being automatically sorted and properly
% "compressed/ranged". e.g., [1], [9], [2], [7], [5], [6] without using
% cite.sty will become [1], [2], [5]--[7], [9] using cite.sty. cite.sty's
% \cite will automatically add leading space, if needed. Use cite.sty's
% noadjust option (cite.sty V3.8 and later) if you want to turn this off
% such as if a citation ever needs to be enclosed in parenthesis.
% cite.sty is already installed on most LaTeX systems. Be sure and use
% version 5.0 (2009-03-20) and later if using hyperref.sty.
% The latest version can be obtained at:
% http://www.ctan.org/pkg/cite
% The documentation is contained in the cite.sty file itself.
%
% Note that some packages require special options to format as the Computer
% Society requires. In particular, Computer Society  papers do not use
% compressed citation ranges as is done in typical IEEE papers
% (e.g., [1]-[4]). Instead, they list every citation separately in order
% (e.g., [1], [2], [3], [4]). To get the latter we need to load the cite
% package with the nocompress option which is supported by cite.sty v4.0
% and later.

% *** GRAPHICS RELATED PACKAGES ***
%
\ifCLASSINFOpdf
  % \usepackage[pdftex]{graphicx}
  % declare the path(s) where your graphic files are
  % \graphicspath{{../pdf/}{../jpeg/}}
  % and their extensions so you won't have to specify these with
  % every instance of \includegraphics
  % \DeclareGraphicsExtensions{.pdf,.jpeg,.png}
\else
  % or other class option (dvipsone, dvipdf, if not using dvips). graphicx
  % will default to the driver specified in the system graphics.cfg if no
  % driver is specified.
  % \usepackage[dvips]{graphicx}
  % declare the path(s) where your graphic files are
  % \graphicspath{{../eps/}}
  % and their extensions so you won't have to specify these with
  % every instance of \includegraphics
  % \DeclareGraphicsExtensions{.eps}
\fi
% graphicx was written by David Carlisle and Sebastian Rahtz. It is
% required if you want graphics, photos, etc. graphicx.sty is already
% installed on most LaTeX systems. The latest version and documentation
% can be obtained at: 
% http://www.ctan.org/pkg/graphicx
% Another good source of documentation is "Using Imported Graphics in
% LaTeX2e" by Keith Reckdahl which can be found at:
% http://www.ctan.org/pkg/epslatex
%
% latex, and pdflatex in dvi mode, support graphics in encapsulated
% postscript (.eps) format. pdflatex in pdf mode supports graphics
% in .pdf, .jpeg, .png and .mps (metapost) formats. Users should ensure
% that all non-photo figures use a vector format (.eps, .pdf, .mps) and
% not a bitmapped formats (.jpeg, .png). The IEEE frowns on bitmapped formats
% which can result in "jaggedy"/blurry rendering of lines and letters as
% well as large increases in file sizes.
%
% You can find documentation about the pdfTeX application at:
% http://www.tug.org/applications/pdftex


% *** MATH PACKAGES ***
%
%\usepackage{amsmath}
% A popular package from the American Mathematical Society that provides
% many useful and powerful commands for dealing with mathematics.
%
% Note that the amsmath package sets \interdisplaylinepenalty to 10000
% thus preventing page breaks from occurring within multiline equations. Use:
%\interdisplaylinepenalty=2500
% after loading amsmath to restore such page breaks as IEEEtran.cls normally
% does. amsmath.sty is already installed on most LaTeX systems. The latest
% version and documentation can be obtained at:
% http://www.ctan.org/pkg/amsmath

% *** SPECIALIZED LIST PACKAGES ***
%
%\usepackage{algorithmic}
% algorithmic.sty was written by Peter Williams and Rogerio Brito.
% This package provides an algorithmic environment fo describing algorithms.
% You can use the algorithmic environment in-text or within a figure
% environment to provide for a floating algorithm. Do NOT use the algorithm
% floating environment provided by algorithm.sty (by the same authors) or
% algorithm2e.sty (by Christophe Fiorio) as the IEEE does not use dedicated
% algorithm float types and packages that provide these will not provide
% correct IEEE style captions. The latest version and documentation of
% algorithmic.sty can be obtained at:
% http://www.ctan.org/pkg/algorithms
% Also of interest may be the (relatively newer and more customizable)
% algorithmicx.sty package by Szasz Janos:
% http://www.ctan.org/pkg/algorithmicx


% *** ALIGNMENT PACKAGES ***
%
%\usepackage{array}
% Frank Mittelbach's and David Carlisle's array.sty patches and improves
% the standard LaTeX2e array and tabular environments to provide better
% appearance and additional user controls. As the default LaTeX2e table
% generation code is lacking to the point of almost being broken with
% respect to the quality of the end results, all users are strongly
% advised to use an enhanced (at the very least that provided by array.sty)
% set of table tools. array.sty is already installed on most systems. The
% latest version and documentation can be obtained at:
% http://www.ctan.org/pkg/array

% IEEEtran contains the IEEEeqnarray family of commands that can be used to
% generate multiline equations as well as matrices, tables, etc., of high
% quality.

% *** SUBFIGURE PACKAGES ***
%\ifCLASSOPTIONcompsoc
%  \usepackage[caption=false,font=footnotesize,labelfont=sf,textfont=sf]{subfig}
%\else
%  \usepackage[caption=false,font=footnotesize]{subfig}
%\fi
% subfig.sty, written by Steven Douglas Cochran, is the modern replacement
% for subfigure.sty, the latter of which is no longer maintained and is
% incompatible with some LaTeX packages including fixltx2e. However,
% subfig.sty requires and automatically loads Axel Sommerfeldt's caption.sty
% which will override IEEEtran.cls' handling of captions and this will result
% in non-IEEE style figure/table captions. To prevent this problem, be sure
% and invoke subfig.sty's "caption=false" package option (available since
% subfig.sty version 1.3, 2005/06/28) as this is will preserve IEEEtran.cls
% handling of captions.
% Note that the Computer Society format requires a sans serif font rather
% than the serif font used in traditional IEEE formatting and thus the need
% to invoke different subfig.sty package options depending on whether
% compsoc mode has been enabled.
%
% The latest version and documentation of subfig.sty can be obtained at:
% http://www.ctan.org/pkg/subfig

% *** FLOAT PACKAGES ***
%
%\usepackage{fixltx2e}
% fixltx2e, the successor to the earlier fix2col.sty, was written by
% Frank Mittelbach and David Carlisle. This package corrects a few problems
% in the LaTeX2e kernel, the most notable of which is that in current
% LaTeX2e releases, the ordering of single and double column floats is not
% guaranteed to be preserved. Thus, an unpatched LaTeX2e can allow a
% single column figure to be placed prior to an earlier double column
% figure.
% Be aware that LaTeX2e kernels dated 2015 and later have fixltx2e.sty's
% corrections already built into the system in which case a warning will
% be issued if an attempt is made to load fixltx2e.sty as it is no longer
% needed.
% The latest version and documentation can be found at:
% http://www.ctan.org/pkg/fixltx2e

%\usepackage{stfloats}
% stfloats.sty was written by Sigitas Tolusis. This package gives LaTeX2e
% the ability to do double column floats at the bottom of the page as well
% as the top. (e.g., "\begin{figure*}[!b]" is not normally possible in
% LaTeX2e). It also provides a command:
%\fnbelowfloat
% to enable the placement of footnotes below bottom floats (the standard
% LaTeX2e kernel puts them above bottom floats). This is an invasive package
% which rewrites many portions of the LaTeX2e float routines. It may not work
% with other packages that modify the LaTeX2e float routines. The latest
% version and documentation can be obtained at:
% http://www.ctan.org/pkg/stfloats
% Do not use the stfloats baselinefloat ability as the IEEE does not allow
% \baselineskip to stretch. Authors submitting work to the IEEE should note
% that the IEEE rarely uses double column equations and that authors should try
% to avoid such use. Do not be tempted to use the cuted.sty or midfloat.sty
% packages (also by Sigitas Tolusis) as the IEEE does not format its papers in
% such ways.
% Do not attempt to use stfloats with fixltx2e as they are incompatible.
% Instead, use Morten Hogholm'a dblfloatfix which combines the features
% of both fixltx2e and stfloats:
%
% \usepackage{dblfloatfix}
% The latest version can be found at:
% http://www.ctan.org/pkg/dblfloatfix

% *** PDF, URL AND HYPERLINK PACKAGES ***
%
%\usepackage{url}
% url.sty was written by Donald Arseneau. It provides better support for
% handling and breaking URLs. url.sty is already installed on most LaTeX
% systems. The latest version and documentation can be obtained at:
% http://www.ctan.org/pkg/url
% Basically, \url{my_url_here}.

% *** Do not adjust lengths that control margins, column widths, etc. ***
% *** Do not use packages that alter fonts (such as pslatex).         ***
% There should be no need to do such things with IEEEtran.cls V1.6 and later.
% (Unless specifically asked to do so by the journal or conference you plan
% to submit to, of course. )

% correct bad hyphenation here
\hyphenation{op-tical net-works semi-conduc-tor}
   
\usepackage{hyperref}
 
\begin{document}
% 
% paper title
% Titles are generally capitalized except for words such as a, an, and, as,
% at, but, by, for, in, nor, of, on, or, the, to and up, which are usually
% not capitalized unless they are the first or last word of the title.
% Linebreaks \\ can be used within to get better formatting as desired.
% Do not put math or special symbols in the title.
\title{In-browser Text Importance Detector\\
{\small \today~-~\currenttime}}


% author names and affiliations
% use a multiple column layout for up to three different
% affiliations
\author{\IEEEauthorblockN{Paulo Ricardo Botelho Oliva}
\IEEEauthorblockA{University of Luxembourg\\
  Email: paulo.botelho.001@student.uni.lu}
\\
{\bf This report has been produced under the supervision of:}\\
\IEEEauthorblockN{Luis Leiva}
\IEEEauthorblockA{University of Luxembourg\\
  Email: luis.leiva@uni.lu}%
}

% conference papers do not typically use \thanks and this command
% is locked out in conference mode. If really needed, such as for
% the acknowledgment of grants, issue a \IEEEoverridecommandlockouts
% after \documentclass

% for over three affiliations, or if they all won't fit within the width
% of the page (and note that there is less available width in this regard for
% compsoc conferences compared to traditional conferences), use this
% alternative format:
% 
%\author{\IEEEauthorblockN{Michael Shell\IEEEauthorrefmark{1},
%Homer Simpson\IEEEauthorrefmark{2},
%James Kirk\IEEEauthorrefmark{3}, 
%Montgomery Scott\IEEEauthorrefmark{3} and
%Eldon Tyrell\IEEEauthorrefmark{4}}
%\IEEEauthorblockA{\IEEEauthorrefmark{1}School of Electrical and Computer Engineering\\
%Georgia Institute of Technology,
%Atlanta, Georgia 30332--0250\\ Email: see http://www.michaelshell.org/contact.html}
%\IEEEauthorblockA{\IEEEauthorrefmark{2}Twentieth Century Fox, Springfield, USA\\
%Email: homer@thesimpsons.com}
%\IEEEauthorblockA{\IEEEauthorrefmark{3}Starfleet Academy, San Francisco, California 96678-2391\\
%Telephone: (800) 555--1212, Fax: (888) 555--1212}
%\IEEEauthorblockA{\IEEEauthorrefmark{4}Tyrell Inc., 123 Replicant Street, Los Angeles, California 90210--4321}}




% use for special paper notices
%\IEEEspecialpapernotice{(Invited Paper)}




% make the title area
\maketitle

%to remove for your report
%\footnote{}

% As a general rule, do not put math, special symbols or citations
% in the abstract
\begin{abstract}
  % TODO: abstract

  \lipsum[1][]
\end{abstract}

% no keywords

% For peer review papers, you can put extra information on the cover
% page as needed:
% \ifCLASSOPTIONpeerreview
% \begin{center} \bfseries EDICS Category: 3-BBND \end{center}
% \fi
%
% For peerreview papers, this IEEEtran command inserts a page break and
% creates the second title. It will be ignored for other modes.
\IEEEpeerreviewmaketitle


\section{Introduction}
% no \IEEEPARstart

% TODO: add the introduction

This Bachelor Semester Project is all about finding important sentences in a text. The main objective of the project is to create a browser extension that highlights the most important information of an article's web page. Moreover, this project also addresses the topic of automatic text summarization, as well as extractive text summarization algorithms.


\section{Project description}
\subsection{Domains}
\subsubsection{Scientific}

% TODO: add the scientific domains

\lipsum[][1-2]

\subsubsection{Technical}

% TODO: add the technical domains

\lipsum[][1-2]

\subsection{Targeted Deliverables}
\label{sec-deliverables}

\subsubsection{Scientific deliverable}

% TODO: add the scientific deliverable description

This BSP's scientific deliverable aims to answer the following research question:\\

\textbf{How can we detect important sentences in a web document?}\\

To help answer this question, we will need to answer the following secondary questions:\\

\textbf{1. How can we automatically detect text blocks on a web page?}\\

\textbf{2. How can we extract relevant sentences within a text block?}\\

To answer our main scientific question, we will investigate the concept of automatic text summarization. Automatic text summarization is the technique of creating a shorter version of a text using a computer program. This shorter version of the text (i.e., the summary) contains the most relevant information of the original text.

Automatic text summarization methods are very much needed to deal with the ever-increasing volume of text data available online, both to help identify relevant information and to consume relevant information more quickly. There are many reasons why one would need automatic text summarization. For example, summaries reduce reading time and should be less biased than summaries made by a human person \cite{torres2014automatic}.

In general, there are two main kinds of text summarization techniques, extractive text summarization and abstractive text summarization. Extractive text summarization consists in selecting the most important parts of a text and extracting them to create a shorter summary of the same text. Abstractive text summarization consists in gathering the main ideas of a text and using those abstract ideas to generate a summary of the text, where the summary isn't just a copy of the old text, but actual new sentences.

Given the scope of this BSP, our scientific deliverable will focus on extractive text summarization. There exist many approaches to extractive text summarization, like statistical approaches, lexical chain based approaches, graph-based approaches, cluster-based approaches, and fuzzy logic based approaches, among others.

We will focus on graph-based text summarization, as it is commonly used by many algorithms to classify sentences by their importance on the source text. For example, Google's PageRank is a graph-based ranking algorithm used by Google to rank web pages \cite{page1999pagerank}. This algorithm is the basis of other well-known graph-based text summarization algorithms like TextRank \cite{mihalcea2004textrank} and LexRank \cite{erkan2004lexrank}.

The graph-based method of text summarization is an unsupervised technique in which sentences or words are scored using a graph, therefore their convenience for this BSP. In a nutshell, the basic goal of graph-based methods is to extract the most relevant sentences from a block of text. Thus, the scientific deliverable will focus on explaining how graph-based text summarization algorithms work.

To answer the secondary questions, the notion of text blocks should be explained. In the context of web pages, a text block is considered a group of paragraphs. Critically, we need to identify the most important text blocks in a web page (the main content) as well as unimportant text blocks, such as texts in the header or the footer of the page. Unimportant texts should not be considered for summarization.

A key challenge is that automatic detection of the main text blocks on a web page is not reliable, since every web page has a different structure and layout. Thus, detecting blocks of text is a hard task. The scientific deliverable of this BSP will study this question more thoroughly. As there does not seem to be a trivial solution, we will try to find the best compromise solution.

\subsubsection{Technical deliverable}

% TODO: add the technical deliverable description

The technical deliverable of this project focuses on creating a browser extension that detects the most important sentences on a specific web page and highlights them, so that the user only needs to read the highlighted sentences instead of the whole web page.

A browser extension is a small software package that adds features and functions to a browser. Extensions are built using well-known web technologies such as HTML, CSS, and JavaScript. Extensions rely on their own set of APIs, which are browser-dependent. For example, they allow developers to change the browser's default behaviours or inject custom stylesheets of JavaScript content before a web page is loaded.

The extension should be built using JavaScript and should work on the Google Chrome browser, given its large market share of 65-70\% \cite{netmarketshare} \cite{statcounter}, without breaking the functionality of existing websites. The extension will automatically highlight the most important sentences of an article or blog post in a specific web page.

In short, the goal of this technical work is to implement text summarization techniques in a browser environment, while creating a simple extension that is easy to use and that works on most websites.

\section{Prerequisites}
% Describe in these sections the main scientific and technical knowledge that is required to be known by you before starting the project.
% Do not describe in details this knowledge but only abstractly. All the content of this section shall not be used, even partially, in the deliverable sections.
% It is important not to include in this section all the knowledge you have been obliged to acquire in order to produce the deliverable. It should only state the knowledge the student possessed before starting the project and that was mandatory to possess to be capable to produce the deliverables. It explicitly defines  the technical and scientific pre-condition for the project. It is also useful to avoid project failures due to over or under complex subjects.

% TODO: looks too short

Before starting the project, a basic understanding of a few topics is required. In particular, a basic understanding of web technologies, such as HTML, CSS, and JavaScript, is required.

\subsection{Scientific prerequisites}

There are no scientific prerequisites for this Bachelor
Semester Project.

\subsection{Technical prerequisites}

The only required technical competency before starting
to work on this Bachelor Semester Project is possessing
a basic amount of knowledge about HTML, CSS and the JavaScript programming language. 

\section{Scientific Deliverable}
\label{sec-sci-production}
\subsection{Requirements}

% TODO: needs more details/content

The scientific deliverable produced in this project consists in answering the question "How can we detect important sentences in a web document?". In this case, a web document is any web page presented to us in the browser while browsing the Internet.

This question has been split into two parts: the first part is about automatic text detection in a web page, and the second part is about extracting the most important sentences from a text block.
Thus, the scientific deliverable requires these two questions to be answered, ultimately leading to a conclusion for the question "How can we detect important sentences in a web document?".


\subsection{Design}
% Provide the necessary and most useful explanations on how those deliverables have been produced.

% TODO: add the scientific design

\subsection{Production}
% Provide descriptions of the deliverables concrete production. It must present part of the deliverable (e.g. source code extracts, scientific work extracts, \ldots) to illustrate and explain its actual production.

% TODO: add the scientific production

% concepts definitions (Extractive Text Summarization, etc.)

% Main question

% Question 1: How can we automatically detect text blocks on a web page?

% Explanation...

% Question 2: How can we extract relevant sentences within a text block?

% Explanation...

% back to main question

% thus, the main question is answered by the two questions above.


\subsection{Assessment}
% Provide any objective elements to assess that your deliverables do or do not satisfy the requirements described above.

% TODO: add the scientific assessment

\section{Technical Deliverable}
\label{sec-tech-production}
\subsection{Requirements}

% TODO: rework this section

The requirements are the characteristics that are required of the completed technical deliverable.
The software developed for the technical deliverable must meet a number of functional and non-functional requirements.

\subsubsection{Functional Requirements}

\begin{itemize}
  \item Gathering the text on a web page.
  \item Generating a summary of the gathered text.
  \item Highlighting the important text on a web page.
  \item Running at least on Google Chrome.
\end{itemize}

\vspace{0.4cm}
\textbf{Gathering the text}

The browser extension must be able to gather the text of any web page.
The extension must contain a function that gathers the text of a web page.
The function must be able to gather the text of any web page.


\vspace{0.4cm}
\textbf{Generating a summary}

The extension must allow the user to generate a summary of the text gathered from the web page.
The summary must be generated using the most important sentences of the text, with the help of an extractive text summarization algorithm.
The user must be able to configure the summary to be generated.
The parameters of the summary must be configurable, depending on the summarization algorithm used.
Such parameters are, for example, the length of the summary generated.

\vspace{0.4cm}
\textbf{Highlighting the important text}

The extension must be able to highlight the important text on the web page, based on the summary generated by the summarization algorithm.
The most important sentences of the text must be highlighted after the user has generated a summary.

\vspace{0.4cm}
\textbf{Running at least on Google Chrome}

The browser extension must be able to run at least on Google Chrome.
Other browsers may be supported, but the extension must be able to run on Google Chrome without errors.


\subsubsection{Non-functional Requirements}
\begin{itemize}
  \item Providing a high-quality summary.
  \item Maintaining the functionality of existing websites.
\end{itemize}

\vspace{0.4cm}
\textbf{Providing a high-quality summary}

The summary generated by the summarization algorithm must be high-quality.
A high-quality summary should allow the user to quickly understand the content of the web page.


\vspace{0.4cm}
\textbf{Maintaining the functionality of existing websites}

The browser extension must not break the functionality of existing websites.
Thus, if the user decides to use the extension on a website, all the website's features must remain functional. 



\subsection{Design}

% TODO: finish the technical design

This section explains the design of the application developed in the technical deliverable.
The software is written using vsCode, although any other text editor would be enough for the task.
The browser extension is written using HTML, CSS and JavaScript in conjunction to make the extension.

\subsubsection{Libraries used}
This project mainly depends on 3 different sets of libraries to work:
\begin{itemize}
  \item Summary.js
  \item textrank.js
  \item naiverank.js
\end{itemize}

% TODO: add a paragraph for each library describing the library and its purpose
% TODO: insert missing ref
Summary.js is "a lightweight paragraph summarizer library" that can be configured to appease to the user's preferences.
It uses the LexRank algorithm to score sentences and gives the best scoring sentences back as a summary.

% TODO: describe these 2
Textrank.js


Naiverank.js


\subsubsection{Project Structure}

An extension is basically a bundled package of different HTML, CSS and JavaScript files.
This extension consists of 2 main parts: the content script and the pop-up.

\vspace{0.4cm}
\textbf{Content Script}

%TODO: describe the content script, its role in the extension and how it interacts with the extension/web page

\vspace{0.4cm}
\textbf{Pop-up}

%TODO: describe the pop-up, its purpose, the files it contains and how it interacts with the extension/web page

\vspace{0.4cm}
\textbf{Manifest file}

%TODO: describe the manifest file


\subsubsection{User Interface}

%TODO: describe the user interface of the extension


% Maybe add a subsection for the highlight functionality?

\subsection{Production}

% TODO: add the technical production


\subsubsection{Project Development}

\subsubsection{Content Script}

% Important function such as the one that gathers the text from the web page and highlighting

\subsubsection{Pop-up}

\subsubsection{Manifest file}


\subsection{Assessment}

% TODO: add the technical assessment

\section*{Acknowledgment}
The author would like to thank the BiCS management and education team for the amazing work done.


\section{Conclusion}

% TODO: add the conclusion

\newpage


\section{Plagiarism statement}

I declare that I am aware of the following facts:
\begin{itemize}
  \item As a student at the University of Luxembourg I must respect the rules of intellectual honesty, in particular not to resort to plagiarism, fraud or any other method that is illegal or contrary to scientific integrity.
  \item My report will be checked for plagiarism and if the plagiarism check is positive, an internal procedure will be started by my tutor. I am advised to request a pre-check by my tutor to avoid any issue.
  \item As declared in the assessment procedure of the University of Luxembourg, plagiarism is committed whenever the source of information used in an assignment, research report, paper or otherwise published/circulated piece of work is not properly acknowledged. In other words, plagiarism is the passing off as one's own the words, ideas or work of another person, without attribution to the author. The omission of such proper acknowledgement amounts to claiming authorship for the work of another person. Plagiarism is committed regardless of the language of the original work used. Plagiarism can be deliberate or accidental.
        Instances of plagiarism include, but are not limited to:
        \begin{enumerate}
          \hbadness=10000
          \item Not putting quotation marks around a quote from another person's work
          \item Pretending to paraphrase while in fact quoting
          \item Citing incorrectly or incompletely
          \item Failing to cite the source of a quoted or paraphrased work
          \item Copying/reproducing sections of another person's work without acknowledging the source
          \item Paraphrasing another person's work without acknowledging the source
          \item Having another person write/author a work for oneself and submitting/publishing it (with permission, with or without compensation) in one's own name ('ghost-writing')
          \item Using another person's unpublished work without attribution and permission ('stealing')
          \item Presenting a piece of work as one's own that contains a high proportion of quoted/copied or paraphrased text (images, graphs, etc.), even if adequately referenced
        \end{enumerate}
        Auto- or self-plagiarism, that is the reproduction of (portions of a) text previously written by the author without citing that text, i.e. passing previously authored text as new, may be regarded as fraud if deemed sufficiently severe.
\end{itemize}


% references section

\newpage
\nocite{bics-bsp-report-template}
\bibliographystyle{IEEEtran}
\bibliography{references}


\newpage
\section{Appendix}


\end{document}


